\documentclass[11pt,pdftex,oneside]{article}

\usepackage[english]{babel}
\usepackage{amsmath,amssymb,amsfonts}
\usepackage[final]{graphicx}
\usepackage[small,bf,hang]{caption}
\usepackage[utf8]{inputenc}
\usepackage[T1]{fontenc}
\usepackage{ae,aecompl}
\usepackage{fancyhdr}
\usepackage{pxfonts}
\usepackage{url}

\addtolength{\textwidth}{2cm}
\addtolength{\hoffset}{-1cm}
\addtolength{\voffset}{0.5cm}
\addtolength{\textheight}{35pt}
\addtolength{\textheight}{1cm}
\addtolength{\topmargin}{-2cm}

\pagestyle{fancy}
\fancyhead{}
\fancyfoot{}
\lhead{\nouppercase{\rightmark}}
\rhead{\nouppercase{\leftmark}}
\renewcommand*{\subsectionmark}[1]{\markboth{\thesubsection\ \ #1}{}}
\fancyhead[LO]{\nouppercase\leftmark}
\fancyhead[RO]{{\bfseries\thepage}}
\addtolength{\headheight}{\baselineskip}

\displaywidowpenalty=4000
\widowpenalty=4000
\clubpenalty=4000

\title{Comparative Programming Languages\\ {\Large Assignment two}}
\author{Ramses De Norre \and Robin Debruyne
        \and Lynn Gyselen \and Mathias Spiessens}
\date{December 16, 2011}

\begin{document}
\maketitle
\thispagestyle{empty}

\section{Features}
Our DSL, which we referred to as Rumadeus, is built as a thin wrapper
around the Adameus query system.
It is an internal DSL written in ruby (hence the name).
We have tried to provide a more readable syntax than the obscure Adameus
queries and added some higher level features such as multi-hop bookings and
reservations, finding the best priced ticket and others.

Our DSL's main client is a REPL on which commands and their parameters can
be given.
To start the REPL, one executes in the {\tt src/} directory the following
command\footnote{We used ruby versions 1.9.2 and 1.9.3 during development,
we cannot guarantee compatibility with other versions because of the
general instability of libraries between major versions and our limited
time frame.}
\begin{quote}
  {\tt ruby -I . ./clients/REPL.rb}
\end{quote}
after which you will be greeted by a prompt
\begin{quote}
  {\tt Rumadeus >}
\end{quote}
A list of available commands and their parameters can be obtained by the
{\tt help} or {\tt h} command.

Below we show and example Rumadeus session which also shows some error
handling and corner cases like an empty result set.
The parameters can all be entered in a natural format and no parentheses
or separators are needed.
Furthermore, data such as dates can be entered in a multitude of formats,
basically any format which ruby's {\tt Date::parse} method
understands.
\begin{samepage}
  \begin{quote}
    \begin{verbatim}
Welcome to Rumadeus!
Type help for a list of commands.

Rumadeus > list_connections 2011-11-05 BRU AFR
(Empty result.)

Rumadeus > list_connections 2012-01-31 TEG AMS
2012-01-31 TEG AMS GWI101 19:30 01:15
2012-01-31 TEG AMS KLM137 18:55 01:05
2012-01-31 TEG AMS DLH169 19:55 01:05

Rumadeus > list_seats GWI101
You did not supply the correct amount of arguments.
Wrong number of arguments (1 for 3).

Rumadeus > list_seats 2012-01-31 GWI101 B
8 175

Rumadeus > list_seats 2012/01/31 GWI101 B
8 175

Rumadeus > list_seats 2012.01.31 GWI101 B
8 175

Rumadeus >
    \end{verbatim}
  \end{quote}
\end{samepage}
Another example that show cases the more high level features of Rumadeus
can be found in the file {\tt report/example\_scenario.txt}, we did not
inline it here because of the large output generated.

A test suite can be found in the {\tt test/} directory and can be run by
executing the following command in the {\tt src/} directory (yes, {\tt
src/}! Not {\tt test/}):
\begin{quote}
  {\tt ruby -I . ../test/ts\_rumadeus.rb}
\end{quote}
note that the Adameus server should be up and running for some of the
tests to succeed.

\section{Cool Features}
In this section we will mention some nice features of Rumadeus which we
would like to highlight.

\subsection{Shortest path}
Given two airports and a maximum number of hops, Rumadeus can calculate
the shortest path in time between these airports.
The maximum number of hops is needed because with the current Adameus
architecture of one query per telnet connection, a full search of all
possible routes was not a feasible option.

If, for example, you want to find the shortest path from Brussels to JFK
with a maximum of two transfers on the 31st of January 2012, you could
enter the query
\begin{quote}
  \begin{verbatim}
shortest_with_stops 31/01/2012-01:15 BRU JFK 2 E
  \end{verbatim}
\end{quote}
the system will then make the following suggestion:
flight GWI098 from BRU to FRA and flight AAL341 from FRA to JFK.
The system takes into account whether there are seats available on the
suggested flights such that only flights with seats available are returned.
The example used can also be found in the aforementioned file with queries.

\subsection{Group bookings over multiple connections}
The system has the ability to hold, book and cancel reservations for a
variable number of persons and a variable number of connections in a
single query.
This is also shown in the example queries.
Queries which handle such \emph{multi-queries} are always denoted with the
keyword {\tt multi}.

\subsection{Help!}
For a full list of available queries type {\tt help} or {\tt h} in our
REPL.
The output of help is generated dynamically from the ruby classes by the
use of reflection and aliases for the query methods to method names with a
common prefix (see further on for more details on this prefix).
Help lists all the functionality provided for external use including the
input parameters.
By giving the parameters meaningful names the user can easily understand
the required input when reasonably familiar with the system.

\subsection{REPL}
The main interface to Rumadeus is a REPL.
We preferred this over a script-based DSL because in the given setting it
seemed more appropriate to have users dynamically enter commands than
having them write up a custom script for every session.

The REPL has a help function and does proper managing of corner cases
(like too many or few parameters, unknown commands, wrong formatting of
parameters) and error handling.

\section{Implementation}

The main issue in developing Rumadeus was figuring out how to do method
dispatch such that users are presented a number of functions that can be
located in multiple source files.
We also wanted to be able to write helper functions in those files without
the need to expose those to the user as well.
Furthermore the user has to provide parameters in a string-only interface
and these all have to be parsed.

To reach these goals, we used ruby's excellent support for dynamic
programming and the possibility to hook right into the method dispatch
mechanisms by using methods as {\tt public\_send} and
{\tt method\_missing}.
In this section we will elaborate a bit on how we used these mechanisms.

\subsection{Calling methods from strings}
To call a method by means of user provided strings, we used the
{\tt public\_send} method which allows one to dynamically call a method
from a string on any object with given parameters, in the form of strings
as well.
The user-accessible methods are furthermore all prefixed by a common
string, which is used to ensure that only those functions are accessible.

\subsection{Dispatch amongst multiple classes}
To have method dispatch amongst multiple classes we used the chain of
command pattern, which was very easy to implement using ruby's
{\tt method\_missing}.
The main code for this mechanism can be found in the {\tt AbstractQuery}
class.
In concreto, every query-providing class extends this subclass and
specifies, through a template method, a delegate to which method dispatch
should go if a given query cannot be found in the current class.
By doing so, a request gets forwarded through our chain of query-providing
classes until it is either handled somewhere or until it reaches the final
class (called {\tt LastResort}) which responds with \emph{unknown
command}.

The chain of command makes it easy to extend the DSL by just hooking a new
class with extra queries into the existing chain, furthermore an end-user
can also easily add in new functionality by e.g.\ monkey patching the
{\tt LastResort} class.

\subsection{Help}
The help functionality is implemented with the same chain of command as
method dispatch and can also mainly be found in {\tt AbstractQuery}.
We used the reflective ruby methods {\tt methods}, {\tt instance\_method}
and {\tt parameters} to find the methods in the current class that start
with our query prefix and then obtain their parameters as well.
These parameters are then formatted to show whether they are regular,
optional or vararg methods, the last two are, respectively, wrapped in
parentheses and prepended with a~{\tt $\displaystyle\ast$}.

By doing this for our entry point of the chain of command and then having
each class concatenate the result of the help for the next element of
the chain, we obtain a list of all available commands and their
parameters.

\subsection{The query classes}
We provide a concise overview of the different classes responsible for
executing queries.
\subsubsection{Actions.rb}

This class handles and parses the queries that hold, book, look up or
cancel a flight in the database.
The return values of the database are stored in objects (found in
the utilities directory) when the query is successful, otherwise a
{\tt Util::ReservationError}, a subclass of ruby's {\tt StandardError}, is
thrown.
The REPL handles these errors, with their specified message, when they are
thrown.

\subsubsection{HLActions.rb}

Advanced actions concerning multiple flights are provided here.
The bookings use a transactional model, this way a roll-back is possible
if any booking fails due to, for instance, concurrent access of multiple
clients.

\subsubsection{Query.rb}

In this class the basic queries provided by the Adameus server are wrapped
in functions.

\subsubsection{HLQuery.rb}

This class provides higher level functionality, for example finding the
shortest path between two airports on a given date.

\section{Ruby Findings}
We summarise our first impressions of the ruby language after this project.

\subsection{Pros}

\begin{itemize}
  \item Intuitive programming
  \item Readable language
  \item Very fast prototyping possible
  \item Open classes allow for great extensibility (albeit not without
    risk)
  \item Method missing and explicit message sending allows to hook
    directly into the message dispatch mechanism
  \item Great support for reflection and meta-programming
\end{itemize}

\subsection{Cons}

\begin{itemize}
  \item Weak typing, tests were necessary to avoid erroneous code
  \item Poor tool support, next to no information (like available methods,
    types, \ldots) available at \emph{develop-time} due to the lack of
    typing and features such as open classes.
  \item Good for quick prototyping, but it seems to us that it would be
    more difficult to manage and maintain bigger projects in a language as
    ruby due to poor refactoring support and the greater risk of wrong
    usage of a method or class due to the absence of type checking and
    proper access privilege enforcement.
\end{itemize}

\end{document}
